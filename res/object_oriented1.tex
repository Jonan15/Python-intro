
\documentclass[10pt]{beamer}
% \usepackage{subfiles}

\usepackage[danish]{babel}
\usepackage[utf8]{inputenc}
\usetheme[progressbar=frametitle]{metropolis}
\usepackage{appendixnumberbeamer}


\usepackage{booktabs}
\usepackage[scale=2]{ccicons}

% \usepackage{pgfplots}
% \usepgfplotslibrary{dateplot}

\definecolor{Orange}{RGB}{229,134,25}

\usepackage{xspace}
\newcommand{\themename}{\textbf{\textsc{metropolis}}\xspace}
\usepackage{listings}


\definecolor{background}{RGB}{226, 226, 226}


\lstset{ 
literate=% 
{Ö}{{\"O}}1 
{Ä}{{\"A}}1 
{Ü}{{\"U}}1 
{ß}{{\ss}}{ 1\negmedspace\,} 
{ü}{{\"u}}1 
{ä}{{\"a}}1 
{ö}{{\"o}}1 
{ø}{{\o}}{1\negmedspace\,} 
{Ø}{{\O}}{1\negthinspace\,\,} 
{å}{{\aa}}{1\negthickspace\,} 
{Å}{{\AA}}{1\negthinspace\;} 
{æ}{{\ae}}{1\negthinspace\;} 
{Æ}{{\AE}}{1\,\,}}

\lstdefinestyle{terminal}{
	language=bash,
	aboveskip=2mm,
	belowskip=2mm,
	showstringspaces=false,
	columns=flexible,
	basicstyle={\small\ttfamily},
	numbers=none,
	numberstyle=\footnotesize,
	commentstyle=\color{black},
	frame=single,
	framesep=2pt,
	breaklines=true,
	breakatwhitespace=false,
	backgroundcolor = \color{background},
	tabsize=2
}


\lstdefinestyle{python}{
	language=Python,
	aboveskip=2mm,
	belowskip=1mm,
	showstringspaces=false,
	columns=flexible,
	numbers=none,
	numberstyle=\footnotesize,
	commentstyle=\ttfamily\color{black},
	frame=single,
	framesep=2pt,
	breaklines=true,
	breakatwhitespace=false,
	backgroundcolor = \color{background},
	tabsize=2
}



\title{Lær Python dag 1 - modul 1}
\subtitle{Introduktion, basis python}
% \date{\today}
\date{}
\author{Steffen Berg Klenow \\Jonas Bamse Andersen}
\institute{Syddansk Universitet}
% \titlegraphic{\hfill\includegraphics[height=1.5cm]{logo.pdf}}


\title{Lær Python dag 4 - modul 1}
\subtitle{Objektorienteret programmering}

\begin{document}

\maketitle

\begin{frame}{Indhold}
  \setbeamertemplate{section in toc}[sections numbered]
  \tableofcontents[hideallsubsections]
\end{frame}


\section{Objektorienteret programmering}
\begin{frame}{Objektorienteret programmering}
	\Large
	"Object-oriented programming (OOP) is a programming paradigm based on the concept of "objects", which may contain data, in the form of fields, often known as attributes; and code, in the form of procedures, often known as methods".\\
	\large
	- Wikipedia
\end{frame}

\begin{frame}{Objektorienteret programmering}
	Objektorienteret programmering (OOP) baseres på koncepted om "objekter". Følgende beskriver et objekt:
	\begin{itemize}
		\item Indeholder værdier (attributes) og kode (methods).
		\item Beskriver et koncept.
		\item Har ansvar for egen kode.
	\end{itemize}
	Objektorienteret programmering er et programmeringsparadigme, altså en måde at programmere og strukturere sin kode på.
\end{frame}

\begin{frame}{Objekt og klasse}
	Lidt to sidder af samme sag...\\
	\bigskip
	\textbf{Klasse}: En brugerdefineret prototype, som definerer et sæt attributter og methoder.\\
	\bigskip
	\textbf{Objekt}: En unik instans af en datastruktur defineret af dens klasse. Et objekt indeholder data.



\end{frame}


\begin{frame}{Eksempler på klasser}
	\begin{itemize}
		\item Køretøj: \\
			attributes: antal hjul, længde, højde\\
			methoder: dyt(), stop(), start()
		\item Person: \\
			attributes: navn, alder, forældre (Person)\\
			methoder: hils(), skift\_navn(), løb()
	\end{itemize}
\end{frame}


\begin{frame}[fragile]{Har vi allerede brugt objekter?}
	\pause
	Vi har fx set et filobjekt:
	\begin{lstlisting}[style=python]
f = open("min_fil.txt", "w")
f.write("Lær python!!!\n")
f.write("IMADA SDU")
f.close()
	\end{lstlisting}
	Her bruger vi metoderne \texttt{write()} og \texttt{close()} på vores filobjekt.\\
\end{frame}


\begin{frame}[fragile]{Har vi allerede brugt objekter?}
		Filobjektet har også attributes fx:
	\begin{itemize}
		\item name: Navnet på filen (string)
		\item closed: Er filen lukket eller ej (boolean)
		\item mode: Hvordan er filen blivet åbnet (string)
	\end{itemize}
	\begin{columns}
		\column{0.5\textwidth}
		\begin{lstlisting}[style=python]
f = open("min_fil.txt", "w")
print(file.name)
print(file.closed)
print(file.mode)
f.close()
		\end{lstlisting}
		\column{0.3\textwidth}
		\begin{lstlisting}[style=python]
min_fil.txt
False
w
		\end{lstlisting}
	\end{columns}
\end{frame}

\section{Klasser og objekter i python}


\begin{frame}[fragile]{Definition af klasse}
	En klasse defineres på følgende vis:
		\begin{lstlisting}[style=python]
class Person:
	def __init__(self, navn, alder):
		self.navn = navn
		self.alder = alder
	\end{lstlisting}
	\texttt{\_\_init\_\_} er en speciel methode som bliver kaldt når et nyt objekt skal laves. Denne methode skal have \emph{self} som første parameter. Dette er en reference til klassen selv og bruges til at få adgang til variabler for klassen.
\end{frame}

\begin{frame}[fragile]{Initialisering}
	Et nyt objekt kan nu laves:
		\begin{lstlisting}[style=python]
person1 = Person("Jens", 24)
person2 = Person("Peter", 23)
	\end{lstlisting}
	Og attributter kan tilgås vha. \texttt{.}-operatoren:
	\begin{lstlisting}[style=python]
print("person1: ", person1.navn, person1.alder)
print("person2: ", person2.navn, person2.alder)
	\end{lstlisting}
	Output:
	\begin{lstlisting}[style=python]
person1:  Jens 24
person2:  Peter 23
	\end{lstlisting}
\end{frame}

\begin{frame}{Hvor kan de bruges?}
	Objekter kan bruges alle steder hvor en normal primitiv type kan bruges, dvs. vi kan fx bruge det til følgende:
	\begin{itemize}
		\item Som et parameter til en funktion.
		\item Som en attribut i et andet objekt.
		\item Som returtype fra en funktion.
	\end{itemize}
\end{frame}

\begin{frame}[fragile]{Eksempel}
	Lad os lave en klasse som gemmer et klokkeslæt:
		\begin{lstlisting}[style=python]
class Tid:
	def __init__(self, timer, minutter, sekunder):
		self.timer = timer
		self.minutter = minutter
		self.sekunder = sekunder
	\end{lstlisting}
	\pause
	Vi kan lave en funktion som printer tiden:
	\begin{lstlisting}[style=python]
def print_tid(t):
	print("%02d:%02d:%02d" % (t.timer, t.minutter, t.sekunder))
	\end{lstlisting}
	\pause
	Lad os teste det:
	\begin{columns}
		\column{0.4\textwidth}
		\begin{lstlisting}[style=python]
min_tid = Tid(2, 3, 23)
print_tid(min_tid)
		\end{lstlisting}
		\column{0.4\textwidth}
		\begin{lstlisting}[style=python]
02:03:23
		\end{lstlisting}
	\end{columns}
\end{frame}

\section{Metoder}


\section{Nedarvning}




\end{document}