\documentclass[main.tex]{subfiles}

\title{Lær Python dag 2 - modul 2}
\subtitle{Strenge}

\begin{document}

\maketitle

%\begin{frame}{Indhold}
%  \setbeamertemplate{section in toc}[sections numbered]
%  \tableofcontents[hideallsubsections]
%\end{frame}

\section{Strenge}
\begin{frame}[fragile]{Disclaimer}
Jeg forsøger at give et overblik over nogle muligheder, men forventer ikke I husker det hele.

\end{frame}


\begin{frame}[fragile]{Strenge - en sekvens af tegn}
	I har set tekststrenge mange gange allerede.
	\begin{lstlisting}[style=python]
name = input("Intast dit navn: ")
print("Hej " + name + "!")
	\end{lstlisting}
\end{frame}

\begin{frame}[fragile]{Strenge - en sekvens af tegn}
	Men strenge kan bruges til alt muligt, f.eks. til at repræsentere en DNA-sekvens eller en bog.
	\begin{lstlisting}[style=python]
dna = "ATTAGCC"
book = "Once upon a time ..."
	\end{lstlisting}
\end{frame}

\begin{frame}[fragile]{Strenge - en sekvens af tegn}
	En streng er på mange måder bare en liste af enkelte tegn.
	\begin{lstlisting}[style=python]
dna = "ATTAGCC"
	\end{lstlisting}
	\begin{tabular}{|l|c|c|c|c|c|c|c|}
		\hline
		Tegn 	& A & T & T & A & G & C & C \\
		\hline
		Index 	& 0 & 1 & 2 & 3 & 4 & 5 & 6 \\
		\hline
	\end{tabular}
\pause

Så derfor kan vi bruge mange af de samme funktioner som vi lærte til lister!
\end{frame}

\begin{frame}[fragile]{Strenge - en sekvens af tegn}
	Indeksering
	\begin{columns}
		\column{0.4\textwidth}
			\begin{lstlisting}[style=python]
dna = "ATTAGCC"
print(dna[1])
			\end{lstlisting}
			\pause
			
		\column{0.4\textwidth}
			\begin{lstlisting}[style=python]
T
			\end{lstlisting}
			
	\end{columns}
	
	\pause
	Slicing
	\begin{columns}
		\column{0.4\textwidth}
			\begin{lstlisting}[style=python]
dna = "ATTAGCC"
print(dna[2:5])
			\end{lstlisting}
		
		\pause
		\column{0.4\textwidth}
			\begin{lstlisting}[style=python]
TAG
			\end{lstlisting}
	\end{columns}
	\pause
	Længde
	\begin{columns}
		\column{0.4\textwidth}
		\begin{lstlisting}[style=python]
dna = "ATTAGCC"
print(len(dna))
		\end{lstlisting}
		
		\pause
		\column{0.4\textwidth}
		\begin{lstlisting}[style=python]
7
		\end{lstlisting}
	\end{columns}
\end{frame}

\begin{frame}[fragile]{Strenge - en sekvens af tegn}
	Gennemløb af liste (iterate)
	\begin{columns}
		\column{0.4\textwidth}
			\begin{lstlisting}[style=python]
dna = "ATTAGCC"
for c in dna:
	print(c)
			\end{lstlisting}
			
		\pause
		\column{0.4\textwidth}
			\begin{lstlisting}[style=python]
A
T
T
A
G
C
C
			\end{lstlisting}
	\end{columns}
\end{frame}

\begin{frame}[fragile]{Strenge - en sekvens af tegn}
	Lister er mutable, hvad med strenge?
	\begin{columns}
		\column{0.4\textwidth}
			\begin{lstlisting}[style=python]
mylist = [1, 2, 3, 4]
mylist[0] = 5
print(mylist)

dna = "ATTAGCC"
dna[0] = "G"
print(dna)
			\end{lstlisting}
		
		\pause
		\column{0.4\textwidth}
			\begin{lstlisting}[style=python]
[5, 2, 3, 4]
Traceback (most recent call last):
	File "test.py", line 6, in <module>
		dna[0] = "G"
TypeError: 'str' object does not support item assignment
			\end{lstlisting}
	\end{columns}
\end{frame}


\begin{frame}[fragile]{Strenge - en sekvens af tegn}
	En løsning på immutability er at bygge nye strenge
		\begin{columns}
		\column{0.4\textwidth}
		\begin{lstlisting}[style=python]
dna = "ATTAGCC"
newdna = "G" + dna[1:]
print(newdna)
		\end{lstlisting}
		
		\pause
		\column{0.4\textwidth}
		\begin{lstlisting}[style=python]
GTTAGCC
		\end{lstlisting}
	\end{columns}
\end{frame}

\begin{frame}[fragile]{Strenge - en sekvens af tegn}
	Hvad med \texttt{.append()}?
	\begin{columns}
		\column{0.4\textwidth}
		\begin{lstlisting}[style=python]
dna = "ATTAGCC"
dna.append("A")
print(dna)
		\end{lstlisting}
		
		\pause
		\column{0.4\textwidth}
		\begin{lstlisting}[style=python]
Traceback (most recent call last):
	File "test.py", line 2, in <module>
		dna.append("A")
AttributeError: 'str' object has no attribute 'append'
		\end{lstlisting}
	\end{columns}
\end{frame}

\begin{frame}[fragile]{Strenge - en sekvens af tegn}
	Heldigvis ved vi at vi kan sammensætte (konkatenere) strenge med "+"
		\begin{columns}
		\column{0.4\textwidth}
		\begin{lstlisting}[style=python]
dna = "ATTAGCC"
newdna = dna + "A"
print(newdna)
		\end{lstlisting}
		
		\pause
		\column{0.4\textwidth}
		\begin{lstlisting}[style=python]
ATTAGCCA
		\end{lstlisting}
	\end{columns}
\end{frame}


\begin{frame}[fragile]{Strenge - en sekvens af tegn}
	Der er også visse funktioner (metoder) som er lavet til strenge, f.eks. \texttt{.lower()} og \texttt{.upper()} som ændrer alle bogstaver til henholdsvis små og store.
	\begin{columns}
		\column{0.4\textwidth}
		\begin{lstlisting}[style=python]
dna = "ATTAGCC"
newdna = dna.lower()
print(newdna)
		\end{lstlisting}
		
		\pause
		\column{0.4\textwidth}
		\begin{lstlisting}[style=python]
attagcc
		\end{lstlisting}
	\end{columns}
\end{frame}

\begin{frame}[fragile]{Strenge - en sekvens af tegn}
	Hvis man er interesseret i at vide hvor i en streng et tegn forekommer, kan man bruge \texttt{.find()}.
	\begin{columns}
		\column{0.4\textwidth}
		\begin{lstlisting}[style=python]
dna = "ATTAGCC"
print(dna.find("G"))
		\end{lstlisting}
		
		\pause
		\column{0.4\textwidth}
		\begin{lstlisting}[style=python]
4
		\end{lstlisting}
	\end{columns}
	\pause
	Hvad hvis der er flere af den samme, eller slet ikke nogen?
	\begin{columns}
		\column{0.4\textwidth}
		\begin{lstlisting}[style=python]
dna = "ATTAGCC"
print(dna.find("A"))
print(dna.find("B"))
		\end{lstlisting}
		
		\pause
		\column{0.4\textwidth}
		\begin{lstlisting}[style=python]
0
-1
		\end{lstlisting}
		Første forekomst og -1
	\end{columns}
\end{frame}

\begin{frame}[fragile]{Strenge - en sekvens af tegn}
	Kan man søge på mere end bare et enkelt tegn?
	\begin{columns}
	\column{0.4\textwidth}
	\begin{lstlisting}[style=python]
dna = "ATTAGCC"
print(dna.find("TAG"))
	\end{lstlisting}
		
	\pause
	\column{0.4\textwidth}
	\begin{lstlisting}[style=python]
2
	\end{lstlisting}
	Ja, svaret er index fra, hvor forekomsten starter.
	\end{columns}	
\end{frame}

\begin{frame}[fragile]{Strenge - en sekvens af tegn}
	Hvis man er ligeglad med HVOR en forekomst er, men bare vil vide OM det forekommer kan man bruge \texttt{in}.
	\begin{columns}
		\column{0.4\textwidth}
		\begin{lstlisting}[style=python]
dna = "ATTAGCC"
print("TAG" in dna)
print("CAT" in dna)
		\end{lstlisting}
		
		\pause
		\column{0.4\textwidth}
		\begin{lstlisting}[style=python]
True
False
		\end{lstlisting}
	\end{columns}
\end{frame}

\begin{frame}[fragile]{Strenge - en sekvens af tegn}
	Man kan inddele en sætning i ord ved at skære den i stykker ved mellemrum og få alle stykkerne i en liste. Til det bruger man \texttt{.split()} funktionen. 
	
	\pause
	Man skal fortælle ved hvilket slags tegn man gerne vil skære (ofte mellemrum).
	\begin{columns}
		\column{0.4\textwidth}
		\begin{lstlisting}[style=python]
s = "Jeg gik mig over sø og land"
l = s.split(" ")
print(l)
print(l[2])
		\end{lstlisting}
		
		\pause
		\column{0.4\textwidth}
		\begin{lstlisting}[style=python]
['Jeg', 'gik', 'mig', 'over', 'sø', 'og', 'land']
mig
		\end{lstlisting}
	\end{columns}
\end{frame}

\begin{frame}[fragile]{Strenge - en sekvens af tegn}
	Det modsatte af at splitte er at sætte sammen, til det bruger man .join(). Den sætter en streng ind mellem alle elementerne i listen. Det bruges ofte til hurtigt at lave en liste af strenge til en enkelt streng, ved at joine med mellemrum.
	\begin{columns}
		\column{0.4\textwidth}
		\begin{lstlisting}[style=python]
l = ['Jeg', 'gik', 'mig', 'over', 'sø', 'og', 'land']
binde = "-".join(l)
print(binde)
		\end{lstlisting}
		
		\pause
		\column{0.4\textwidth}
		\begin{lstlisting}[style=python]
Jeg-gik-mig-over-sø-og-land
		\end{lstlisting}
	\end{columns}
\end{frame}

\begin{frame}[fragile]{Strenge - en sekvens af tegn}
	Man kan også erstatte bogstaver med \texttt{.replace()}. Her skal man fortælle, hvad der skal ændres og hvad det skal ændres til. 
	
	\pause
	Kan f.eks. bruges til at fjerne mellemrum.
	
	\begin{columns}
		\column{0.4\textwidth}
		\begin{lstlisting}[style=python]
s = "Jeg gik mig over sø og land"
s2 = s.replace(" ", "")		# Erstatter mellemrum med ingenting = fjerner mellemrum
print(s2)
		\end{lstlisting}
		
		\pause
		\column{0.4\textwidth}
		\begin{lstlisting}[style=python]
Jeggikmigoversøogland
		\end{lstlisting}
	\end{columns}
\end{frame}

\begin{frame}[fragile]{Strenge - en sekvens af tegn}
	Et eksempel på brug af nogle af de ting vi nu har lært kunne være en funktion som givet to ord (strenge), printer alle bogstaver der forekommer i begge ord.
	\begin{columns}
		\column{0.45\textwidth}
		\begin{lstlisting}[style=python]
def in_both(word1, word2):
	#code?
		\end{lstlisting}
		
		\pause
		\column{0.4\textwidth}
		
	\end{columns}
\end{frame}

\begin{frame}[fragile]{Strenge - en sekvens af tegn}
	Et eksempel på brug af nogle af de ting vi nu har lært kunne være en funktion som givet to ord (strenge), printer alle bogstaver der forekommer i begge ord.
	\begin{columns}
		\column{0.6\textwidth}
		\begin{lstlisting}[style=python]
def in_both(word1, word2):
    for letter in word1:
        if (letter in word2):
            print(letter)
            
in_both("APPLE", "PEN")
		\end{lstlisting}

		\pause
		\column{0.3\textwidth}
		\begin{lstlisting}[style=python]
P
P
E
		\end{lstlisting}
	\end{columns}
\end{frame}



% \begin{frame}{Metropolis titleformats}
%     \themename supports 4 different titleformats:
%     \begin{itemize}
%         \item Regular
%         \item \textsc{Smallcaps}
%         \item \textsc{allsmallcaps}
%         \item ALLCAPS
%     \end{itemize}
%     They can either be set at once for every title type or individually.
% \end{frame}

% {
%     \metroset{titleformat frame=smallcaps}
% \begin{frame}{Small caps}
%     This frame uses the \texttt{smallcaps} titleformat.

%     \begin{alertblock}{Potential Problems}
%         Be aware, that not every font supports small caps. If for example you typeset your presentation with pdfTeX and the Computer Modern Sans Serif font, every text in smallcaps will be typeset with the Computer Modern Serif font instead.
%     \end{alertblock}
% \end{frame}
% }

% {
% \metroset{titleformat frame=allsmallcaps}
% \begin{frame}{All small caps}
%     This frame uses the \texttt{allsmallcaps} titleformat.

%     \begin{alertblock}{Potential problems}
%         As this titleformat also uses smallcaps you face the same problems as with the \texttt{smallcaps} titleformat. Additionally this format can cause some other problems. Please refer to the documentation if you consider using it.

%         As a rule of thumb: Just use it for plaintext-only titles.
%     \end{alertblock}
% \end{frame}
% }

% {
% \metroset{titleformat frame=allcaps}
% \begin{frame}{All caps}
%     This frame uses the \texttt{allcaps} titleformat.

%     \begin{alertblock}{Potential Problems}
%         This titleformat is not as problematic as the \texttt{allsmallcaps} format, but basically suffers from the same deficiencies. So please have a look at the documentation if you want to use it.
%     \end{alertblock}
% \end{frame}
% }

% \section{Elements}

% \begin{frame}[fragile]{Typography}
%       \begin{verbatim}The theme provides sensible defaults to
% \emph{emphasize} text, \alert{accent} parts
% or show \textbf{bold} results.\end{verbatim}

%   \begin{center}becomes\end{center}

%   The theme provides sensible defaults to \emph{emphasize} text,
%   \alert{accent} parts or show \textbf{bold} results.
% \end{frame}

% \begin{frame}{Font feature test}
%   \begin{itemize}
%     \item Regular
%     \item \textit{Italic}
%     \item \textsc{SmallCaps}
%     \item \textbf{Bold}
%     \item \textbf{\textit{Bold Italic}}
%     \item \textbf{\textsc{Bold SmallCaps}}
%     \item \texttt{Monospace}
%     \item \texttt{\textit{Monospace Italic}}
%     \item \texttt{\textbf{Monospace Bold}}
%     \item \texttt{\textbf{\textit{Monospace Bold Italic}}}
%   \end{itemize}
% \end{frame}

% \begin{frame}{Lists}
%   \begin{columns}[T,onlytextwidth]
%     \column{0.33\textwidth}
%       Items
%       \begin{itemize}
%         \item Milk \item Eggs \item Potatos
%       \end{itemize}
%
%     \column{0.33\textwidth}
%       Enumerations
%       \begin{enumerate}
%         \item First, \item Second and \item Last.
%       \end{enumerate}
%
%     \column{0.33\textwidth}
%       Descriptions
%       \begin{description}
%         \item[PowerPoint] Meeh. \item[Beamer] Yeeeha.
%       \end{description}
%   \end{columns}
% \end{frame}
% \begin{frame}{Animation}
%   \begin{itemize}[<+- | alert@+>]
%     \item \alert<4>{This is\only<4>{ really} important}
%     \item Now this
%     \item And now this
%   \end{itemize}
% \end{frame}
% \begin{frame}{Figures}
%   \begin{figure}
%     \newcounter{density}
%     \setcounter{density}{20}
%     \begin{tikzpicture}
%       \def\couleur{alerted text.fg}
%       \path[coordinate] (0,0)  coordinate(A)
%                   ++( 90:5cm) coordinate(B)
%                   ++(0:5cm) coordinate(C)
%                   ++(-90:5cm) coordinate(D);
%       \draw[fill=\couleur!\thedensity] (A) -- (B) -- (C) --(D) -- cycle;
%       \foreach \x in {1,...,40}{%
%           \pgfmathsetcounter{density}{\thedensity+20}
%           \setcounter{density}{\thedensity}
%           \path[coordinate] coordinate(X) at (A){};
%           \path[coordinate] (A) -- (B) coordinate[pos=.10](A)
%                               -- (C) coordinate[pos=.10](B)
%                               -- (D) coordinate[pos=.10](C)
%                               -- (X) coordinate[pos=.10](D);
%           \draw[fill=\couleur!\thedensity] (A)--(B)--(C)-- (D) -- cycle;
%       }
%     \end{tikzpicture}
%     \caption{Rotated square from
%     \href{http://www.texample.net/tikz/examples/rotated-polygons/}{texample.net}.}
%   \end{figure}
% \end{frame}
% \begin{frame}{Tables}
%   \begin{table}
%     \caption{Largest cities in the world (source: Wikipedia)}
%     \begin{tabular}{lr}
%       \toprule
%       City & Population\\
%       \midrule
%       Mexico City & 20,116,842\\
%       Shanghai & 19,210,000\\
%       Peking & 15,796,450\\
%       Istanbul & 14,160,467\\
%       \bottomrule
%     \end{tabular}
%   \end{table}
% \end{frame}
% \begin{frame}{Blocks}
%   Three different block environments are pre-defined and may be styled with an
%   optional background color.

%   \begin{columns}[T,onlytextwidth]
%     \column{0.4\textwidth}
%       \begin{block}{Default}
%         Block content.
%       \end{block}

%       \begin{alertblock}{Alert}
%         Block content.
%       \end{alertblock}

%       \begin{exampleblock}{Example}
%         Block content.
%       \end{exampleblock}

%     \column{0.4\textwidth}

%       \metroset{block=fill}

%       \begin{block}{Default}
%         Block content.
%       \end{block}

%       \begin{alertblock}{Alert}
%         Block content.
%       \end{alertblock}

%       \begin{exampleblock}{Example}
%         Block content.
%       \end{exampleblock}

%   \end{columns}
% \end{frame}
% \begin{frame}{Math}
%   \begin{equation*}
%     e = \lim_{n\to \infty} \left(1 + \frac{1}{n}\right)^n
%   \end{equation*}
% \end{frame}
% \begin{frame}{Line plots}
%   \begin{figure}
%     \begin{tikzpicture}
%       \begin{axis}[
%         mlineplot,
%         width=0.9\textwidth,
%         height=6cm,
%       ]

%         \addplot {sin(deg(x))};
%         \addplot+[samples=100] {sin(deg(2*x))};

%       \end{axis}
%     \end{tikzpicture}
%   \end{figure}
% \end{frame}
% \begin{frame}{Bar charts}
%   \begin{figure}
%     \begin{tikzpicture}
%       \begin{axis}[
%         mbarplot,
%         xlabel={Foo},
%         ylabel={Bar},
%         width=0.9\textwidth,
%         height=6cm,
%       ]

%       \addplot plot coordinates {(1, 20) (2, 25) (3, 22.4) (4, 12.4)};
%       \addplot plot coordinates {(1, 18) (2, 24) (3, 23.5) (4, 13.2)};
%       \addplot plot coordinates {(1, 10) (2, 19) (3, 25) (4, 15.2)};

%       \legend{lorem, ipsum, dolor}

%       \end{axis}
%     \end{tikzpicture}
%   \end{figure}
% \end{frame}
% \begin{frame}{Quotes}
%   \begin{quote}
%     Veni, Vidi, Vici
%   \end{quote}
% \end{frame}

% {%
% \setbeamertemplate{frame footer}{My custom footer}
% \begin{frame}[fragile]{Frame footer}
%     \themename defines a custom beamer template to add a text to the footer. It can be set via
%     \begin{verbatim}\setbeamertemplate{frame footer}{My custom footer}\end{verbatim}
% \end{frame}
% }

% \begin{frame}{References}
%   Some references to showcase [allowframebreaks] \cite{knuth92,ConcreteMath,Simpson,Er01,greenwade93}
% \end{frame}

% \section{Conclusion}

% \begin{frame}{Summary}

%   Get the source of this theme and the demo presentation from

%   \begin{center}\url{github.com/matze/mtheme}\end{center}

%   The theme \emph{itself} is licensed under a
%   \href{http://creativecommons.org/licenses/by-sa/4.0/}{Creative Commons
%   Attribution-ShareAlike 4.0 International License}.

%   \begin{center}\ccbysa\end{center}

% \end{frame}

% {\setbeamercolor{palette primary}{fg=black, bg=Orange}
% \begin{frame}[standout]
%   Questions?
% \end{frame}
% }

% \appendix

% \begin{frame}[fragile]{Backup slides}
%   Sometimes, it is useful to add slides at the end of your presentation to
%   refer to during audience questions.

%   The best way to do this is to include the \verb|appendixnumberbeamer|
%   package in your preamble and call \verb|\appendix| before your backup slides.

%   \themename will automatically turn off slide numbering and progress bars for
%   slides in the appendix.
% \end{frame}

% \begin{frame}[allowframebreaks]{References}

%   \bibliography{demo}
%   \bibliographystyle{abbrv}

% \end{frame}

\end{document}