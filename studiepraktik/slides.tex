\documentclass[10pt]{beamer}

\usepackage[danish]{babel}
\usepackage[utf8]{inputenc}
\usetheme[progressbar=frametitle]{metropolis}
\usepackage{appendixnumberbeamer}
\usepackage{booktabs}

\definecolor{Orange}{RGB}{229,134,25}
\usepackage{xspace}
\newcommand{\themename}{\textbf{\textsc{metropolis}}\xspace}
\usepackage{listings}

\definecolor{background}{RGB}{226, 226, 226}

\lstset{ 
literate=% 
{Ö}{{\"O}}1 
{Ä}{{\"A}}1 
{Ü}{{\"U}}1 
{ß}{{\ss}}{ 1\negmedspace\,} 
{ü}{{\"u}}1 
{ä}{{\"a}}1 
{ö}{{\"o}}1 
{ø}{{\o}}{1\negmedspace\,} 
{Ø}{{\O}}{1\negthinspace\,\,} 
{å}{{\aa}}{1\negthickspace\,} 
{Å}{{\AA}}{1\negthinspace\;} 
{æ}{{\ae}}{1\negthinspace\;} 
{Æ}{{\AE}}{1\,\,}}

\lstdefinestyle{terminal}{
	language=bash,
	aboveskip=2mm,
	belowskip=2mm,
	showstringspaces=false,
	columns=flexible,
	basicstyle={\small\ttfamily},
	numbers=none,
	numberstyle=\footnotesize,
	commentstyle=\color{black},
	frame=single,
	framesep=2pt,
	breaklines=true,
	breakatwhitespace=false,
	backgroundcolor = \color{background},
	tabsize=2
}


\lstdefinestyle{python}{
	language=Python,
	aboveskip=2mm,
	belowskip=1mm,
	showstringspaces=false,
	columns=flexible,
	numbers=none,
	numberstyle=\footnotesize,
	commentstyle=\ttfamily\color{black},
	frame=single,
	framesep=2pt,
	breaklines=true,
	breakatwhitespace=false,
	backgroundcolor = \color{background},
	tabsize=2
}



\title{Introduktion til Programmering i Python}
\subtitle{Studiepraktik 2019}
\date{}
\author{Jonas Bamse Andersen}
\institute{Syddansk Universitet}


\begin{document}

%%%%%%%%%%%%%%%%%%%%%%%%%%%%%%%%%%%%%%%%%%%%%%%%%%%%%%%%%%%%%%%%%%%%%%%%%%%

\maketitle

%%%%%%%%%%%%%%%%%%%%%%%%%%%%%%%%%%%%%%%%%%%%%%%%%%%%%%%%%%%%%%%%%%%%%%%%%%%


\begin{frame}{Indhold}
  \setbeamertemplate{section in toc}[sections numbered]
  \tableofcontents[hideallsubsections]
\end{frame}

%%%%%%%%%%%%%%%%%%%%%%%%%%%%%%%%%%%%%%%%%%%%%%%%%%%%%%%%%%%%%%%%%%%%%%%%%%%

\section{Programmering generelt}

%%%%%%%%%%%%%%%%%%%%%%%%%%%%%%%%%%%%%%%%%%%%%%%%%%%%%%%%%%%%%%%%%%%%%%%%%%%

\begin{frame}{Programmering}
Få en computer til at udføre ønskede handlinger.

\medskip

Computer = desktop, laptop, smartphone, vaskemaskine-controller$\dots$

\medskip

Handlinger: Databehandling og interaktion med hardware.

\medskip

Eksempler:

\begin{itemize}
 \item Tekstbehandling
 \item WWW
 \item Databaser
 \item Computerspil
 \item Styring af andre maskiner (industrimaskiner, måneraketter, $\dots$).
 \item Regneopgaver, videnskabelige udregninger
\end{itemize}
\end{frame}

%%%%%%%%%%%%%%%%%%%%%%%%%%%%%%%%%%%%%%%%%%%%%%%%%%%%%%%%%%%%%%%%%%%%%%%%%%%

\begin{frame}[fragile]{Programmeringssprog}
\begin{center}
	\begin{tabular}{c|c}
		\textbf{Mennesker}						& 				\textbf{Computere}			\\\hline
		Uformel beskrivelse 					& 				Entydige kommandoer			\\
		Naturlige sprog 						& 				Fast format
	\end{tabular}
\end{center}

\begin{center}
	\begin{tabular}{c|c}
		\textbf{Højniveausprog}												& 			\textbf{Maskinkode}				\\\hline
		\texttt{if}$\dots$ \texttt{then}$\dots$ \texttt{else}$\dots$		&			01110111 						\\
		Tæt på engelsk(?) 													& 			Binære koder					\\
		Bedre overblik 														& 			Dårligt overblik				\\
		Sværere at lave fejl 												& 			Nemt at lave fejl				\\
		Hurtig programmering 												& 			Langsom programmering			\\
	\end{tabular}
\end{center}

\bigskip
Oversætter-program: Højniveau sprog $\rightarrow$ maskinkode.
\end{frame}

\begin{frame}{Problemløsning}
I alle programmeringssprog foregår problemløsning/programmering således:

\begin{enumerate}
	\item Inddel problemet i mindre/simplere del-problemer.
	
	\item Inddel de mindre del-problemer i endnu mindre del-problemer.
	
	\item ... 
	
	\item Indtil du kan løse del-problemerne "nemt".
\end{enumerate}

\end{frame}

%%%%%%%%%%%%%%%%%%%%%%%%%%%%%%%%%%%%%%%%%%%%%%%%%%%%%%%%%%%%%%%%%%%%%%%%%%%

\begin{frame}{Programmeringssprog}

De fleste programmeringssprog er af den ``imperative'' type, dvs. er baseret på dataceller (kasser) hvis indhold kan ændres undervejs.

\medskip
Eksempler:

\alert{Java, C, C++, C\#, Pascal, Delphi, Basic, Fortran, Cobol, Ada, Perl,
Python, Ruby, $\dots$
}

Essensen i disse er den samme. Den gennemgås på de følgende sider. Vi
bruger Python som eksempel.

\end{frame}

%%%%%%%%%%%%%%%%%%%%%%%%%%%%%%%%%%%%%%%%%%%%%%%%%%%%%%%%%%%%%%%%%%%%%%%%%%%

\section{Programmering i Python}

%%%%%%%%%%%%%%%%%%%%%%%%%%%%%%%%%%%%%%%%%%%%%%%%%%%%%%%%%%%%%%%%%%%%%%%%%%%

\begin{frame}{Programmering}
	Hurtig start:
	
	\begin{enumerate}
		\item \alert{Åbn} \texttt{repl.it/X3G} og klik `Start Now'.
		\item \alert{Skriv} kode i tekstfeltet.
		\item \alert{Udfør} dit program ved at klikke på `Run' øverst.
	\end{enumerate}

	Prøv tingene af undervejs. Vær nysgerrige.
\end{frame}

\begin{frame}{Grundlæggende elementer}
Data og ændring af data:
\begin{enumerate}
	\item Typer
	\item Variable
	\item Operatorer
\end{enumerate}
Rækkefølge af udførsel af kommandoer:
\begin{enumerate}
	\setcounter{enumi}{4}
	\item Sekvens
	\item Gentagelse
	\item Betinget udførsel
	\item Modularisering
\end{enumerate}

Hvert sprog har desuden en \alert{syntaks} (regler for opbygning af kode),
som kan variere noget.
\end{frame}

%%%%%%%%%%%%%%%%%%%%%%%%%%%%%%%%%%%%%%%%%%%%%%%%%%%%%%%%%%%%%%%%%%%%%%%%%%%

\begin{frame}[fragile]{Eksempler}
	\begin{lstlisting}[style=python]
# The very basic example of Python: Hello World
print("Hello World")
	\end{lstlisting}
	\medskip
	\begin{lstlisting}[style=python]
forbrug = int(input("Hvad er dit elforbrug?"))
pris = 1020 + 1.65 * forbrug
print("Din elregning er:")
print(pris)
	\end{lstlisting}
\end{frame}

%%%%%%%%%%%%%%%%%%%%%%%%%%%%%%%%%%%%%%%%%%%%%%%%%%%%%%%%%%%%%%%%%%%%%%%%%%%

\begin{frame}{1. Typer}

Al data i et program har en \alert{type}.

\medskip
Nogle typer i Python:
\medskip

\begin{center}
	\begin{tabular}{ll}
		\hline
		Datatype						&		Eksempel 	\\ \hline \hline
		String (tekst)					&		"Hej"		\\
		Integer (heltal)				&		42			\\
		Float (kommatal)				&		42.0		\\
		Boolean	(sandhedsværdi)			&		True		\\
	\end{tabular}
\end{center}
\end{frame}

%%%%%%%%%%%%%%%%%%%%%%%%%%%%%%%%%%%%%%%%%%%%%%%%%%%%%%%%%%%%%%%%%%%%%%%%%%%

\begin{frame}[fragile]{2. Variable}
Variabel = Navngiven beholder med data af en bestemt type.

\medskip
Variabel oprettes, når de bruges første gang.\\
Generelt:
\begin{lstlisting}[style=python]
<variabelnavn> = <værdi>
\end{lstlisting}

Eksempler:
\begin{lstlisting}[style=python]
counter = 27
greeting = "Hello World!"
temperature = 44.7
\end{lstlisting}


Variable tildeles værdier med lighedstegn (\texttt{=}):

Bemærk at et variabelnavn på \alert{højre} side henter værdien gemt i
variablen, mens det på \alert{venstre} side gemmer en ny værdi i variablen.

\end{frame}

%%%%%%%%%%%%%%%%%%%%%%%%%%%%%%%%%%%%%%%%%%%%%%%%%%%%%%%%%%%%%%%%%%%%%%%%%%%

\begin{frame}{3. Operatorer: Nye data fra gamle}
Almindelig matematik fungerer nogenlunde som det plejer:
\begin{center}
	\begin{tabular}{lllr}
		\hline
		Operator			& 		Beskrivelse								&		Eksempel		&		Resultat	\\ \hline \hline
		$+$					&		Læg to operander sammen					&		$40 + 2$		&		42			\\
		$-$					&		Træk to operander fra hinanden			&		$50 - 8$		&		42			\\
		$*$					&		Gang to operander 						&		$6 * 7$			&		42			\\
		$/$					&		Division mellem to operander			&		$12.6 / 3$		&		4.2			\\
	\end{tabular}
\end{center}
\end{frame}

%%%%%%%%%%%%%%%%%%%%%%%%%%%%%%%%%%%%%%%%%%%%%%%%%%%%%%%%%%%%%%%%%%%%%%%%%%%

\begin{frame}{3. Operatorer: Nye data fra gamle}
Derudover har vi operatorer som giver en sandhedsværdi:
\begin{center}
	\begin{tabular}{llll}
		\hline
		Operator			& 		Beskrivelse								&		Eksempel		&		Resultat		\\ \hline \hline
		$==$				&		Lig med									&		$42 == 3$		&		False			\\
		$!=$				&		Ikke lig med							&		$42$ != $3$		&		True			\\
		$<$					&		Mindre end		 						&		$6 < 7$			&		True			\\
		$<=$				&		Mindre end eller lig med				&		$6 <= 6$		&		True			\\
		$>$					&		Større end								&		$21 > 5$		&		True			\\
		$>=$				&		Større end eller lig med				&		$21 >= 5$		&		True			\\
	\end{tabular}
\end{center}

\begin{center}
	\begin{tabular}{llll}
		\hline
		Operator			& 		Beskrivelse								&		Eksempel					&		Resultat		\\ \hline \hline
		\textit{and}				&		Boolsk og								&		$(4 <= 6)$ \textit{and} $(3 < 2)$		&		False			\\
		\textit{or}				&		Boolsk eller							&		$(4 <= 6)$ \textit{or} $(3 < 2)$		&		True			\\
		\textit{not}				&		Bolsk ikke		 						&		\textit{not}$(4 < 6)$				&		False			\\
	\end{tabular}
\end{center}
\end{frame}

%%%%%%%%%%%%%%%%%%%%%%%%%%%%%%%%%%%%%%%%%%%%%%%%%%%%%%%%%%%%%%%%%%%%%%%%%%%

\begin{frame}[fragile]{4. Sekvens}

Som udgangspunkt udføres kommandoer én efter én i rækkefølge:

\begin{lstlisting}[style=python]
a = 7
b = a * a
a = 8
b = b + 1
\end{lstlisting}

Hvad ligger der i \texttt{a} og \texttt{b}?
\pause
$$a = 8$$
$$b = 50$$
\end{frame}

%%%%%%%%%%%%%%%%%%%%%%%%%%%%%%%%%%%%%%%%%%%%%%%%%%%%%%%%%%%%%%%%%%%%%%%%%%%

\begin{frame}[fragile]{5. Gentagelse}

Generel \texttt{while}-kommando:

\begin{lstlisting}[style=python]
while <sand/falsk betingelse>:
	<instruktion>
	<instruktion>
	<instruktion>
	...
\end{lstlisting}
Kun den indrykkede kode gentages.

Eksempel (find kvadratroden af 25):

\begin{lstlisting}[style=python]
r = 0
while r * r < 25:
	r = r + 1
print("The square root of 25 is:")
print(r)
\end{lstlisting}

\end{frame}

%%%%%%%%%%%%%%%%%%%%%%%%%%%%%%%%%%%%%%%%%%%%%%%%%%%%%%%%%%%%%%%%%%%%%%%%%%%

\begin{frame}[fragile]{6. Betinget udførsel}

Generel \texttt{if-else}-kommando:

\begin{lstlisting}[style=python]
if <sand/falsk-betingelse>:
	Kode, der udføres, hvis betingelsen er sand
else:
	Kode, der udføres, hvis betingelsen er falsk
\end{lstlisting}
Kun den indrykkede kode udføres.\\
\medskip
Eksempel:

\begin{lstlisting}[style=python]
if a > b:
	print("a is larger")
else:
	print("b is at least as large")
\end{lstlisting}
\end{frame}

%%%%%%%%%%%%%%%%%%%%%%%%%%%%%%%%%%%%%%%%%%%%%%%%%%%%%%%%%%%%%%%%%%%%%%%%%%%

\begin{frame}[fragile]{7. Modularisering}
Metoder (funktion, rutine, procedure): \alert{opdeling} og \alert{genbrug} af kode.

\bigskip
Eksempel:

\begin{lstlisting}[style=python]
def regning(forbrug):
	pris = 1020 + 1.65 * forbrug
	print("Hvis dit forbrug er:")
	print(forbrug)
	print("skal du betale")
	print(pris)

regning(40)
regning(50)
\end{lstlisting}
\end{frame}

%%%%%%%%%%%%%%%%%%%%%%%%%%%%%%%%%%%%%%%%%%%%%%%%%%%%%%%%%%%%%%%%%%%%%%%%%%%

\begin{frame}{Python-biblioteket}

\bigskip

Python kommer med et meget stort bibliotek af færdiglavede metoder.
De bruges til eksempelvis webudvikling, forskning og hobbyprojekter.

\bigskip
Se for eksempel:

\begin{center}
	\texttt{https://wiki.python.org/moin/UsefulModules}
\end{center}
\end{frame}

%%%%%%%%%%%%%%%%%%%%%%%%%%%%%%%%%%%%%%%%%%%%%%%%%%%%%%%%%%%%%%%%%%%%%%%%%%%

\begin{frame}[fragile]{Input og output}

Udskriv til brugeren:

\begin{lstlisting}[style=python]
print(Det der skal skrives)
\end{lstlisting}
Vi kan både udskrive konkrete tal og strenge, men også variabler!

\bigskip

Læs fra tastatur (\texttt{skriv int()} omkring input, hvis du vil læse et tal):
\begin{lstlisting}[style=python]
navn = input("Hvad hedder du?")
forbrug = int(input("Hvad er dit elforbrug?"))
\end{lstlisting}
\end{frame}

%%%%%%%%%%%%%%%%%%%%%%%%%%%%%%%%%%%%%%%%%%%%%%%%%%%%%%%%%%%%%%%%%%%%%%%%%%%

\begin{frame}[fragile]{Eksempler}
	\begin{lstlisting}[style=python]
# The very basic example of Python: Hello World
print("Hello World")
	\end{lstlisting}
	\medskip
	\begin{lstlisting}[style=python]
forbrug = int(input("Hvad er dit elforbrug?"))
pris = 1020 + 1.65 * forbrug
print("Din elregning er:")
print(pris)
	\end{lstlisting}
\end{frame}

%%%%%%%%%%%%%%%%%%%%%%%%%%%%%%%%%%%%%%%%%%%%%%%%%%%%%%%%%%%%%%%%%%%%%%%%%%%

\begin{frame}{Programmering}

\begin{enumerate}
\item \alert{Åbn} \texttt{https://repl.it/X3G} og klik `Start Now'.
\item \alert{Skriv} kode i tekstfeltet.
\item \alert{Udfør} dit program ved at klikke på `Run' øverst.
\end{enumerate}
\end{frame}

%%%%%%%%%%%%%%%%%%%%%%%%%%%%%%%%%%%%%%%%%%%%%%%%%%%%%%%%%%%%%%%%%%%%%%%%%%%
\end{document}
